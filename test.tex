\documentclass[a5paper, fontsize=12pt]{scrbook}
\usepackage[program=lilypond, tmpdir=tmp_ly, pass-fonts, debug]{lyluatex}
\usepackage{multicol}
\usepackage{listings}

\title{}
\author{}

% Si vous voulez changer globalement la taille des partitions.
% Autrement, elle sera automatiquement adaptée à la taille de police.
%\lysetoption{staffsize}{20.5}

\begin{document}

\musicxmlfile{xml/Faur Reve Sample.mxl}

% The environment commands that will be used for verbatim.
\lysetverbenv{\begin{lstlisting}[language=TeX]}{\end{lstlisting}}

% It's possible to wrap `ly` (or `lilypond`) within another environment.
\newenvironment{verbly}{%
\begin{lilypond}[verbatim]
}{%
\end{lilypond}
}

\begin{verbly}
a b c
d e f
\end{verbly}

% Import d'une partition externe. La commande \includely est strictement équivalente.
\lilypondfile[line-width=250pt, insert=fullpage, label=test]{ly/Ave Maria De Lourdes/Ave Maria De Lourdes}

\pageref{ly_test}

% Saisie directe d'un fragment musical.
\begin{lilypond}[staffsize=16]
\version "2.16"
\language "français"

\header {
  tagline = ""
  composer = ""
}

MetriqueArmure = {
  \tempo 4=70
  \time 3/4
  \key do \major
}

italique = { \override Score . LyricText #'font-shape = #'italic }

roman = { \override Score . LyricText #'font-shape = #'roman }

MusiqueTheme = \relative do' {
  \partial 4 do4
  fa4 fa la
  fa4 fa la
  sol4 sol la8[( sol])
  fa2 do4
  fa4 fa la
  fa4 fa la
  sol4 sol la8[( sol])
  fa2 \breathe fa4^"Refrain"
  sib2 sib4
  la2 la4
  sol4 sol sol
  do2 la4
  sib2 sib4
  la2 la4
  sol4 sol la8[( sol])
  fa2. \bar "|."
}

Paroles = \lyricmode {
  L'heure é -- tait ve -- nu -- e,
  où l'ai -- rain sa -- cré,
  de sa voix con -- nu -- e,
  an -- non -- çait l'A -- ve.

  \italique
  A -- ve, a -- ve, a -- ve Ma -- rí -- a_!
  A -- ve, a -- ve, a -- ve Ma -- rí -- a_!
}

\score{
  <<
    \new Staff <<
      \set Staff.midiInstrument = "flute"
      \set Staff.autoBeaming = ##f
      \new Voice = "theme" {
        \override Score.PaperColumn #'keep-inside-line = ##t
        \MetriqueArmure
        \MusiqueTheme
      }
    >>
    \new Lyrics \lyricsto theme {
      \Paroles
    }
  >>
  \layout{}
  \midi{}
}

\end{lilypond}

\begin{lilypond}[staffsize=16, insert=fullpage]
\version "2.16"
\language "français"

\header {
  tagline = ""
  composer = ""
}

MetriqueArmure = {
  \tempo 4=70
  \time 3/4
  \key do \major
}

italique = { \override Score . LyricText #'font-shape = #'italic }

roman = { \override Score . LyricText #'font-shape = #'roman }

MusiqueTheme = \relative do' {
  \partial 4 do4
  fa4 fa la
  fa4 fa la
  sol4 sol la8[( sol])
  fa2 do4
  fa4 fa la
  fa4 fa la
  sol4 sol la8[( sol])
  fa2 \breathe fa4^"Refrain"
  sib2 sib4
  la2 la4
  sol4 sol sol
  do2 la4
  sib2 sib4
  la2 la4
  sol4 sol la8[( sol])
  fa2. \bar "|."
}

Paroles = \lyricmode {
  L'heure é -- tait ve -- nu -- e,
  où l'ai -- rain sa -- cré,
  de sa voix con -- nu -- e,
  an -- non -- çait l'A -- ve.

  \italique
  A -- ve, a -- ve, a -- ve Ma -- rí -- a_!
  A -- ve, a -- ve, a -- ve Ma -- rí -- a_!
}

\score{
  <<
    \new Staff <<
      \set Staff.midiInstrument = "flute"
      \set Staff.autoBeaming = ##f
      \new Voice = "theme" {
        \override Score.PaperColumn #'keep-inside-line = ##t
        \MetriqueArmure
        \MusiqueTheme
      }
    >>
    \new Lyrics \lyricsto theme {
      \Paroles
    }
  >>
  \layout{}
  \midi{}
}

\end{lilypond}

\medskip
L'environnement \texttt{lilypond} :
\begin{lilypond}[relative, verbatim]
{ c d e f g a b c }
\addlyrics{do ré mi fa sol la si do.}
\end{lilypond}

\medskip
\noindent se comporte mieux que la commande du même nom :

\lilypond[relative, verbatim]{
{ c d e f g a b c }
\addlyrics{do ré mi fa sol la si do.}
}

\clearpage
Ce fragment \fbox{\lilypond[label=test2]{a' b' c''}} sera écrit plus petit en note de bas de
page\footnote{\lilypond[relative=2]{a b c}}.

\newpage

Test pour alignement.

\lilypondfile[insert=fullpage, staffsize=16]{ly/Beethoven/opus-18-1.ly}

\end{document}
