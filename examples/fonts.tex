\defaultfontfeatures{Ligatures=TeX,Numbers=OldStyle,Scale=MatchLowercase}
\setmainfont{Linux Libertine O}
\setsansfont[BoldFont={Linux Biolinum O Bold}]{Linux Biolinum O}
\setmonofont{Inconsolata}

\section*{Font Handling}

To demonstrate the font handling features of \lyluatex\ we will repeatedly
include the following score from an external file.  It includes roman (lyrics,
instrument name), sans (rehearsal mark), and mono (tempo) text, first using
LilyPond's built-in default fonts.

\lilypondfile[verbatim]{fonts}

\bigskip
The current document uses \option{fontspec} to set roman font to \emph{Linux
Libertine O}, sans font to \emph{Linux Biolinum O}, and mono font to
\emph{Inconsolata}. So if you compile this document yourself and don't have
these fonts installed you will receive unexpected results.

\subsection*{Passing Document Fonts to Score}

With \option{pass-fonts} the currently active font families for roman, sans, and
mono fonts are passed to LilyPond in order to achieve the most coherent
appearance between text and music.

\bigskip

\lilypondfile[pass-fonts]{fonts}

\bigskip
Note that LilyPond loads fonts differently than \LaTeX\ and can only make use of
fonts installed as system fonts, fonts that are only installed through a \LaTeX\
distribution are not accessible to it. That means that if the document fonts are
not installed system-wide (e.\,g. the default fonts) LilyPond will use rather
ugly fallback fonts. This can't be demonstrated here but the section about
explicitly setting font families will include an example.

The inherent problem of fallback fonts, especially with \LaTeX's default
settings, is the reason \option{pass-fonts} is inactive by default. But the
general recommendation is to set \option{pass-fonts} as package option if the
text document uses fonts that are available to LilyPond.

\bigskip

\sffamily \option{current-font-as-main} will use the font that is
\emph{currently} used for typesetting as LilyPond's main (roman) font. This can
make sure that the score's main font (and roman is usually the font used most in
scores) matches the surrounding text. Note that this might produce surprising
behaviour if it is not clear that the current font has changed, and it will
effectively suppress the original roman font from the score if the current font
is one of the two others. Additionally this \emph{may} introduce an
inconsistency not between the score and the surrounding text but between
different scores in a document. For all these reasons the option is by default
set to \texttt{false}.

\bigskip
\lilypondfile[pass-fonts,current-font-as-main]{fonts}

\subsection*{Setting Score Fonts Explicitly}

With \option{rmfamily}, \option{sffamily}, and \option{ttfamily} specific
families can be set to arbitrary fonts, independently from the text document.
For the following score \option{ttfamily=\{TeXGyre Adventor\}} is
used.\footnote{Note that this font (which is included in TeXLive) has to be
installed if you want to successfully compile this document.} Note that this
implicitly sets \option{pass-fonts=true}, and \emph{Linux Libertine O} and
\emph{Linux Biolinum O} are used from the text document.

\bigskip
\lilypondfile[ttfamily={TeXGyre Adventor}]{fonts}

\highlight{NOTE:} when \option{rmfamily} is set explicitly
\option{current-font-as-main} is forced to \texttt{false} to ensure that the
roman font is actually used. The next score sets \option{rmfamily=\{TeXGyre
Adventor\}} and \option{current-font-as-main}, and despite the current font still being \cmd{sffamily}
\emph{Adventor} is used as the score's main font:

\bigskip
\lilypondfile[current-font-as-main,rmfamily={TeXGyre Adventor}]{fonts}

\subsection*{LilyPond's Font Fallback}

If unavailable fonts are set in a LilyPond document they will \emph{silently} be
replaced with fallback fonts that tend to cause ugly results. This will be shown
by setting \option{rmfamily=FantasyFontOne}, \option{sffamily=FantasyFontTwo},
and \option{tfamily=FantasyFontThree}:

\bigskip
\lilypondfile[rmfamily=FantasyFontOne,%
sffamily=FantasyFontTwo,%
ttfamily=FantasyFontThree]{fonts}

This can happen in several contexts: apart from compiling the document on a
different computer where the used fonts are missing it is most likely to occur
with the \option{pass-fonts} option, when the text document uses internal
\LaTeX\ fonts. Note in particular that this may happen implicitly when only one
family is specified explicitly with an option and the other families are passed
from the text document.
