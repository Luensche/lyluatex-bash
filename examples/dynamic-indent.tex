\documentclass{scrartcl}
\usepackage{lyluatex}
\newcommand{\postLilyPondExample}{\par\bigskip\hrule\par\bigskip}
\begin{document}

\section*{Indent Handling}

This document demonstrates the use of \texttt{indent} and \texttt{autoindent},
partially in combination with \texttt{print-only}.

\texttt{indent=2cm} indents the first line, but if the resulting score contains
only one system this indent is suppressed (issuing a warning on the console):

\begin{lilypond}[indent=2cm]
  \set Staff.instrumentName = "Violin"
  \repeat unfold 12 { c' d' e' d' }
\end{lilypond}

\begin{lilypond}[indent=2cm]
  {
    \set Staff.instrumentName = "Violin"
    c' d' e' d'
  }
\end{lilypond}

If the output of a score which contains more than one system is limited to the
first system using \texttt{print-only=1} then the indent is removed but the
score is recompiled to ensure a full-length system. The following score shows
the two-system score from above (with \texttt{indent=1cm}), limited to its first
system:

\begin{lilypond}[indent=1cm,print-only=1]
  \set Staff.instrumentName = "Violin"
  \repeat unfold 12 { c' d' e' d' }
\end{lilypond}

Note that this behaviour also applies when \texttt{print-only} causes the first
system to be printed at another position, e.g. with \texttt{print-only={3,1,2}}.
In this case the indent of the first system is suppressed in order to avoid a
“hole”. Of course this is a corner case, but might be useful when a score
consists of separate entities (examples, exercises) per system.

\begin{lilypond}[indent=1cm,print-only={3,1,2},max-protrusion=0.5cm]
  \repeat unfold 25 { c' d' e' d' }
\end{lilypond}

If a protrusion limit has been set with \texttt{max-protrusion=0.5cm} and the
score exceeds that limit in spite of \texttt{indent=1cm} then the whole score
will appropriately be narrowed:

\begin{lilypond}[indent=1cm,max-protrusion=0.5cm]
  \set Staff.instrumentName = "Violin I. and II."
  \repeat unfold 11 { c' d' e' d' }
\end{lilypond}


This doesn't really look good because the indentation of the second system
wouldn't have been necessary since only the first system exceeds the protrusion
limit. The solution to this situation is the option \texttt{autoindent} which
handles the indentation \emph{automatically} and set the indent to a value that
will make the \emph{first} system fit into the protrusion limit and leave the
remaining systems unchanged:

\begin{lilypond}[autoindent=true,max-protrusion=0cm]
  \set Staff.instrumentName = "Violin I. and II."
  \repeat unfold 11 { c' d' e' d' }
\end{lilypond}


However, if the protrusion limit is not only exceeded by the \emph{first} system
(which should be the typical case due to the instrument name) \texttt{lyluatex}
will deal with the situation by narrowing the \emph{whole} score by the
appropriate amount and adjusting the indent of the first system so all systems
will just fit into the protrusion limit:

\begin{lilypond}[autoindent=true,max-protrusion=0.5cm]
  \set Staff.instrumentName = "Violin I. and II."
  \set Staff.shortInstrumentName = "Violin I/II"
  \repeat unfold 11 { c' d' e' d' }
\end{lilypond}

There is one special case to be mentioned. As described above the indent is
deactivated if the first system of a score is printed at a later position.
However, if this score will exceed the left protrusion limit \texttt{autoindent}
will be automatically activated to avoid having the \emph{whole} score narrowed:

\begin{lilypond}[indent=1cm,print-only={3,1,2},max-protrusion=0.5cm]
  \set Staff.instrumentName = "Violin"
  \repeat unfold 25 { c' d' e' d' }
\end{lilypond}


The handling of indent suppression may require up to four compilations of the
score, but these are handled automatically, and the resulting intermediate
stages of the score are cached just like the scores actually used in the
document.

The \texttt{autoindent} option is active by default but will be deactivated if
\texttt{indent} is set explicitly. It has to be noted that this option will add
more LilyPond compilations and therefore compilation time. But it will only
apply and be executed if the score exceeds the protrusion limit, so it can only
occur in circumstances where multiple LilyPond runs are expected anyway.

\paragraph{Known Issues:}
If \texttt{autoindent} is implicitly set like in the following example, and
\texttt{max-protrusion=0cm} then the indented first system doesn't correctly
refer to the text margin (\texttt{=0cm}) but to the staff symbol of the
surrounding systems.

\begin{lilypond}[indent=1cm,print-only={3,1,2},max-protrusion=0cm]
  \set Staff.instrumentName = "Violin"
  \repeat unfold 25 { c' d' e' d' }
\end{lilypond}


\end{document}
