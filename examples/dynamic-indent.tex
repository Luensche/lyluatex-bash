\documentclass{scrartcl}
\usepackage{lyluatex}
\newcommand{\postLilyPondExample}{\par\bigskip\hrule\par\bigskip}
\begin{document}

\section*{Indent Handling}

This document demonstrates the use of \texttt{indent} and \texttt{autoindent},
partially in combination with \texttt{print-only}.

\texttt{indent=1cm} indents the first line, but if the resulting score contains
only one system this indent is suppressed (issuing a warning on the console):

\begin{lilypond}[indent=1cm]
  \repeat unfold 12 { c' d' e' d' }
\end{lilypond}

\begin{lilypond}[indent=1cm]
  { c' d' e' d' }
\end{lilypond}

If the output of a score containing more than one system is limited to the first
system using \texttt{print-only=1} then the indent is removed but the score is
recompiled to ensure a full-length system. The following score is the first
system of the first two-system score:

\textbf{This is broken:} The indent should be suppressed. For
this \emph{two} compilations would have been necessary instead of just one.

\begin{lilypond}[indent=1cm,print-only=1]
  \repeat unfold 12 { c' d' e' d' }
\end{lilypond}

Note that this behaviour also applies when \texttt{print-only} causes the first
system to be printed at another position, e.g. with \texttt{print-only={3,1,2}}.
In this case the indent of the first system is suppressed in order to avoid a
“hole”. Of course this is a corner case, but might be useful when a score
consists of separate entities (examples, exercises) per system.

\textbf{Note:} this works correctly, so obviously there's something wrong with
the entry conditional checking whether indent suppression should be applied.

\begin{lilypond}[indent=1cm,print-only={3,1,2}]
  \repeat unfold 25 { c' d' e' d' }
\end{lilypond}


If a protrusion limit has been set with \texttt{max-protrusion=0.5cm} and the
score exceeds that limit in spite of an indent then the whole score will
appropriately be indented:

\begin{lilypond}[indent=1cm,max-protrusion=0.5cm]
  \set Staff.instrumentName = "Violin I. and II."
  \repeat unfold 11 { c' d' e' d' }
\end{lilypond}

This doesn't look very good because the indentation of the consecutive systems
isn't actually necessary if only the first systems has that excessive
protrusion. To improve that situation the option \texttt{autoindent} will add an
automatic indent so only the first system is sufficiently indented:

\textbf{This is broken:} autoindent doesn't seem to make a difference.

\begin{lilypond}[autoindent,max-protrusion=0.5cm]
  \set Staff.instrumentName = "Violin I. and II."
  \repeat unfold 11 { c' d' e' d' }
\end{lilypond}

However, if the protrusion limit is not only exceeded by the \emph{first} system
(which should be the typical case due to the instrument name) \texttt{lyluatex}
will deal with the situation by indenting the \emph{whole} score by the
appropriate amount and adjusting the indent of the first system accordingly:

\textbf{This is broken:} autoindent doesn't seem to make a difference.

\begin{lilypond}[autoindent,max-protrusion=0.5cm]
  \set Staff.instrumentName = "Violin I. and II."
  \set Staff.shortInstrumentName = "Violin I/II"
  \repeat unfold 11 { c' d' e' d' }
\end{lilypond}


The handling of indent suppression may require up to four compilations of the
score, but these are handled automatically, and the resulting intermediate
stages of the score are cached just like the scores actually used in the
document.

\end{document}
