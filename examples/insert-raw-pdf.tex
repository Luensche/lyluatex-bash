\documentclass{lyluatexexample}
\usepackage{pgffor}
\begin{document}


\section*{Wrapping Raw PDF Filenames}

With the \option{raw-pdf} option it is possible to create wrapping commands that
circumvent \lyluatex's layout considerations by working with the raw PDF
filename of the generated score. This is especially useful for developing
packages or personal class and style files. For this scores generated with
\option{raw-pdf} define a command \cmd{lyscore} that can be used in the wrapping
commands or environments.

All examples in this document could also be realized using “default” \lyluatex\
without \option{raw-pdf}, but they are intended to show how this low-level
access can be used to retrieve the information from the generated score in order
to build custom versions of commands that don't have to adhere to \lyluatex's
pre-built strategies of including the score in the document

The easiest way to use a “raw” score is to simply access \cmd{lyscore} in a
command and pass it to an \cmd{includegraphics} macro:

\begin{verbatim}
  \newcommand\lilyinline[2][]{%
  \lily[raw-pdf,%
  insert=bare-inline,%
  inline-staffsize=8,%
  hpadding=0.25ex,#1]{
  \omit Stem
  #2}%
  \includegraphics{\lyscore{}}%
  }
\end{verbatim}

\newcommand\lilyinline[2][]{%
\lily[raw-pdf,insert=bare-inline,inline-staffsize=8,hpadding=0.25ex,#1]{
  \omit Stem
  #2}%
\includegraphics{\lyscore{}}%
}

This basically is a way to provide pre-configured commands. In this case
\lilyinline{ c'8 d' c' d'} it is used to pre-configure an inline
type, entered as \verb+\lilyinline{ c'8 d' c' d'}+.


\bigskip \cmd{lyscore} takes one mandatory argument which can be empty -- as in
the example above --, receive a number, one of the keywords \texttt{nsystems}
and \texttt{hoffset}, or any of the score's options. If passed a number it will
return the filename of the N-th system. With \texttt{nsystems} the number of
systems in the generated score will be returned, while \texttt{hoffset}
generates the code that shifts the score to the left to accommodate protrusion.

The following example takes an optional argument with options that are passed to
\lyluatex, and one mandatory argument which expects the system to be used. It
prints the given system centered in a figure and uses the file name as the
caption and makes use of the score's \texttt{label}. Figure \ref{centered} shows
the centering of a short fragment, figure \ref{fifth} the selection of the fifth
system from a larger score.

\begin{verbatim}
  \newenvironment{centeredlilypondsystem}[2][]{%
  \def\usesystem{#2}
  \begin{figure}
  \begin{center}
    \begin{lilypond}[raw-pdf,#1]%
  }{%
    \end{lilypond}
    \includegraphics{\lyscore{\usesystem}}
    \caption{\lyscore{\usesystem}.pdf}
    \label{\lyscore{label}}
  \end{center}
  \end{figure}
  }

  \begin{centeredlilypondsystem}[label=centered]{1}
  c'1 d' e'
  \end{centeredlilypondsystem}

  \begin{centeredlilypondsystem}[label=fifth]{5}
  \repeat unfold 8 { c'1 \break }
  \end{centeredlilypondsystem}
\end{verbatim}

\newenvironment{centeredlilypondsystem}[2][]{%
\def\usesystem{#2}
\begin{figure}
\begin{center}
  \begin{lilypond}[raw-pdf,#1]%
}{%
  \end{lilypond}
  \includegraphics{\lyscore{\usesystem}}
  \caption{\lyscore{\usesystem}.pdf}
  \label{\lyscore{label}}
\end{center}
\end{figure}
}

\begin{centeredlilypondsystem}[label=centered]{1}
c'1 d' e'
\end{centeredlilypondsystem}

\begin{centeredlilypondsystem}[label=fifth]{5}
\repeat unfold 8 { c'1 \break }
\end{centeredlilypondsystem}


Finally there's an example showing how to iterate over the systems of a score
using \cmd{foreach} from the \option{pgffor} package. It iterates over all the
systems in the given score, prints them using the protrusion adjustment seen
before, and if the system is the third it prints this information, otherwise
just a line break:

\begin{verbatim}
\newcommand\myforlily[2][]{%
\lily[insert=systems,raw-pdf,#1]{#2}%
\foreach \n in {1,...,\lyscore{nsystems}}%
    {\noindent\hspace*{\lyscore{hoffset}}\includegraphics{\lyscore{\n}}%
    \ifthenelse{\equal{\n}{3}}{\par Third system\par}{\\}
    }%
}

\myforlily[staffsize=24]{
\set Staff.instrumentName = "Vl. "
\repeat unfold 4 { c'1 \break } }
\end{verbatim}

\newcommand\myforlily[2][]{%
\lily[insert=systems,raw-pdf,#1]{#2}%
\foreach \n in {1,...,\lyscore{nsystems}}%
    {\noindent\hspace*{\lyscore{hoffset}}\includegraphics{\lyscore{\n}}%
    \ifthenelse{\equal{\n}{3}}{\par\bigskip Third system\par\bigskip}{\\}
    }%
}

\myforlily[staffsize=24]{
\set Staff.instrumentName = "Vl. "
\repeat unfold 4 { c'1 \break } }

\end{document}
