\documentclass{scrartcl}
\usepackage{blindtext}
\usepackage[includepaths={./, ../ly/}]{lyluatex}

\begin{document}

The default behaviour of \texttt{lyluatex} is to align the staff symbol to the
text width and have any protruding elements (left and right) protrude into the
margin.

\begin{lilypond}[nofragment,
%max-left-protrusion=0in,
%max-right-protrusion=10cm,
%indent=1cm,%
print-only=1%
%quote%
]
{
  \set Staff.instrumentName = "Violin 1"
  \shape #'((0 . 0)(0 . 0)(3 . 0)(10 . 0)) Tie
  c'1 ~ \break c'
}
\end{lilypond}

This protrusion is limited to \texttt{max-left-protrusion} and
\texttt{max-right-protrusion}.  This setting only takes effect when the
protruding elements exceed the option value.  In the following score values of
1cm are set, and the score is shortened in order not to exceed the limit:

\begin{lilypond}[nofragment,
max-left-protrusion=1cm,
max-right-protrusion=1cm,
print-only=1%
]
{
  \set Staff.instrumentName = "Violin 1"
  \shape #'((0 . 0)(0 . 0)(3 . 0)(10 . 0)) Tie
  c'1 ~ \break c'
}
\end{lilypond}

This is no explicit indentation but a dynamic limit, like a “fence”. When the
protruding elements are narrower than the limit nothing happens -- in the
following score the limits are still set to 1cm but the elements are shortened
in the score itself.  Note how the staff symbol now aligns again:

\begin{lilypond}[nofragment,
  max-left-protrusion=1cm,
  max-right-protrusion=1cm,
  print-only=1%
]
{
  \set Staff.instrumentName = "Vl."
  \shape #'((0 . 0)(0 . 0)(3 . 0)(2 . 0)) Tie
  c'1 ~ \break c'
}
\end{lilypond}

\bigskip
The shortening is achieved by calculating the possible line width and
recompiling the score with LilyPond.  Note that this is \emph{not} the result of
a scaling of the image file -- the staff size is identical.


\bigskip
By default, there's no limit, so long protruding elements could be cut off:

\begin{lilypond}[nofragment,
print-only=1%
]
{
  \set Staff.instrumentName = "Violin 1 with a comment"
  \shape #'((0 . 0)(0 . 0)(3 . 0)(25 . 0)) Tie
  c'1 ~ \break c'
}
\end{lilypond}

Note that while such a long instrument name isn't common musically, it could be
convenient as a label or margin text.

\bigskip

When used in combination with the \texttt{quote} option the reference is still the staff symbol.  The following three scores have the same quote and maximum protrusion settings: the first doesn't have protruding elements, the second is within the limits and the third applies shortening.

\begin{lilypond}[nofragment,
max-left-protrusion=0.6cm,
max-right-protrusion=0.6cm,
print-only=1,
quote%
]
{
  c'1 ~ \break c'
}
\end{lilypond}

\begin{lilypond}[nofragment,
max-left-protrusion=0.6cm,
max-right-protrusion=0.6cm,
print-only=1,
quote%
]
{
  \set Staff.instrumentName = "Vl."
  \shape #'((0 . 0)(0 . 0)(3 . 0)(2 . 0)) Tie
  c'1 ~ \break c'
}
\end{lilypond}

\begin{lilypond}[nofragment,
max-left-protrusion=0.6cm,
max-right-protrusion=0.6cm,
print-only=1,
quote%
]
{
  \set Staff.instrumentName = "Violin 1"
  \shape #'((0 . 0)(0 . 0)(3 . 0)(8 . 0)) Tie
  c'1 ~ \break c'
}
\end{lilypond}

It is possible to apply \emph{negative} limits, essentially to the effect of a
dynamic indentation. This is different from using the \texttt{quote} option
since it limits the \emph{total} extent of the score rather than specifying the
indentation of the \emph{staff symbol}:

\begin{lilypond}[nofragment,
max-left-protrusion=-1.5cm,
max-right-protrusion=-1.5cm,
print-only=1,
]
{
  \set Staff.instrumentName = "Violin 1"
  \shape #'((0 . 0)(0 . 0)(3 . 0)(6 . 0)) Tie
  c'1 ~ \break c'
}
\end{lilypond}

\begin{lilypond}[nofragment,
max-left-protrusion=-1.5cm,
max-right-protrusion=-1.5cm,
print-only=1,
]
{
  c'1 ~ \break c'
}
\end{lilypond}

\end{document}
