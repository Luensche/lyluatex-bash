\documentclass{scrartcl}
\usepackage{lyluatex}
\newcommand{\postLilyPondExample}{\par\bigskip\hrule\par\bigskip}
\begin{document}

\section*{System Selection and Indent Suppression}

This document demonstrates the use of \texttt{print-only} and \texttt{indent}.
With \texttt{print-only} the output can be limited to a selection of systems or
pages in arbitrary order.  \texttt{lyluatex} suppresses a given indent if either
only the first (or single) system is printed or the first systems happens to be
printed not at the first position.  If necessary one or several recompilation(s)
are triggered automatically.

\bigskip The following score has an indent of 1cm but only a single system.
Therefore the indent is deactivated and the score shifted to the left.  Since
the score itself has only one system no recompilation is required because the
score is ragged-right anyway.

\begin{lilypond}[indent=1cm]
  { c' d' e' d' }
\end{lilypond}

The following score has an indent of 1cm set and \texttt{print-only=3-1},
therefore the indented first system wouldn't be at the top of the score. Since
simply shifting the system to the left would create a “hole” on the right side
the score is automatically recompiled without indentation.

\begin{lilypond}[print-only=3-1,indent=1cm]
  \repeat unfold 28 { c' d' e' d' }
\end{lilypond}

The following score has the same properties as the previous ones but
additionally a \texttt{max-protrusion} limit of 0.5cm and a protruding
instrument name.  This results in the situation that normally the whole score
would be shortened to fit in the maximum protrusion because the indent is
automatically deactivated:

\begin{lilypond}[print-only=2-1,max-protrusion=0.5cm]
  \set Staff.instrumentName = "Violin"
  \repeat unfold 15 { c' d' e' d' }
\end{lilypond}

However, since an indent is given \texttt{lyluatex} tries to fix the protrusion
issue first by reintroducing an indent up to the given value.  This is because
in a majority of cases the protrusion limit will be be exceeded by instrument
names in the first system (which is what indent basically is used for).  In
this score this is sufficient, so only the first system is indented and not the
whole score.

\begin{lilypond}[autoindent, indent=1cm,max-protrusion=0.5cm]
  \set Staff.instrumentName = "Violin"
  \repeat unfold 15 { c' d' e' d' }
\end{lilypond}

If the protrusion is so wide that it would exceed the allowed indent the whole
score will be narrowed by the necessary additional amount:

\begin{lilypond}[print-only=2-1,indent=10cm,max-protrusion=0.5cm]
  \set Staff.instrumentName = "Violin I. and II."
  \repeat unfold 15 { c' d' e' d' }
\end{lilypond}

Finally it is possible that after applying the dynamic indent there still is a
protrusion overflow caused by other systems than the first one.  In this case
the score is narrowed according to the protrusion and the dynamic indent reduced
accordingly:

\begin{lilypond}[print-only=2-1,indent=1cm,max-protrusion=0.5cm]
  \set Staff.instrumentName = "Violin 1"
  \set Staff.shortInstrumentName = "Violin"
  \repeat unfold 15 { c' d' e' d' }
\end{lilypond}



The handling of indent suppression may require up to four compilations of the
score, but these are handled automatically, and the resulting intermediate
stages of the score are cached just like the scores actually used in the
document.

\end{document}
